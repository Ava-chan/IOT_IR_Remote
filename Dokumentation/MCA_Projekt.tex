%allgemeine Formatangaben
\documentclass[
 a4paper, 										% Papierformat
 12pt,												% Schriftgröße
 ngerman, 										% für Umlaute, Silbentrennung etc.
 titlepage,										% es wird eine Titelseite verwendet
 oneside, 										% einseitiges Dokument
 captions=nooneline,					% einzeilige Gleitobjekttitel ohne Sonderbehandlung wie mehrzeilige Gleitobjekttitel behandeln
 numbers=noenddot,						% Überschriften-??Nummerierung ohne Punkt am Ende
 parskip=half,									% zwischen Absätzen wird eine halbe Zeile eingefügt
 ]{scrartcl}


\input{Header/Pakete}					% einbinden der verwendeten Latex-Pakete
\newcommand{\qq}[1]{\glqq{#1\grqq{}}} %Gänsefüßchen
\newcommand{\idx}[1]{#1\index{#1} } % Index drucken und schreiben

\onehalfspacing 							% 1,5facher Zeilenabstand

\definecolor{InterneLinkfarbe}{rgb}{0.1,0.1,0.3} 	% Farbliche Absetzung von externen Links
\definecolor{ExterneLinkfarbe}{rgb}{0.1,0.1,0.7}	% Farbliche Absetzung von internen Links

% Einstellungen für Fußnoten:
%\captionsetup{font=footnotesize,labelfont=sc,singlelinecheck=true,margin={5mm,5mm}}

% Quellenangaben Stil
\bibliographystyle{alphadin}

%Ausschluss von Schusterjungen
\clubpenalty = 10000
%Ausschluss von Hurenkindern
\widowpenalty = 10000


% Beispiel für eine Listings-Codeumbebungen
% Bei mehreren Definitionen empfielt sich das auslagern in eine externe Datei
%{C++}
%\lstset{
%	frame=tb,
%	framesep=5pt,
%	basicstyle=\footnotesize\ttfamily,
%	showstringspaces=false,
%	keywordstyle=\ttfamily\bfseries %\color{CadetBlue},
%	identifierstyle=\ttfamily,
%	stringstyle=\ttfamily %\color{OliveGreen},
%	%commentstyle=\color{GrayBlue},
%	rulecolor=\color{Gray},
%	xleftmargin=5pt,
%	xrightmargin=5pt,
%	aboveskip=\bigskipamount,
%	belowskip=\bigskipamount
%} 

%Den Punkt am Ende jeder Beschreibung deaktivieren
%\renewcommand*{\glspostdescription}{}

\setlength{\cftfignumwidth}{3em} %Befehl von tocloft
\setlength{\cfttabnumwidth}{3em}
%Counter verhalten verändern
%\counterwithin{figure}{section}
%\counterwithin{table}{section}

%Glossar-Befehle anschalten
\makeindex
%\makeglossaries
%\glsenablehyper

\apptocmd{\UrlBreaks}{\do\f\do\m}{}{}
			
\usepackage{tikz}			
\usetikzlibrary{shapes,arrows} %PAP
\begin{document}


\title{Entwicklung eines IR-Empfänger und -Sender mit dem ESP8266}
\subtitle{Dokumentation: Mikrocontroller-Anwendungs-Projekt}
\author{Eike Florian Petersen und Michael Horn} 
\date{\today}

\maketitle

\tableofcontents										% Inhaltsverzeichnis
\pagebreak
\listoffigures											% Abbildungsverzeichnis
\pagebreak
%\listoftables											% Tabellenverzeichnis
%\pagebreak

\section{Einleitung}
\subsection{Ausgangspunkt}
Der Ausgangspunkt des Projektes ist die Absicht eine Infrarotfernbedingung mit Hilfe eines Mikrocontrollers zu realisieren. 
Im Rahmen des aktuellen Hype um \acs{IoT} stellt diese Anforderung für uns eine interessante Möglichkeit dar, bisher nicht \acs{IoT}-fähige, über Infrarotfernbedienung gesteuerten Geräten, diese Funktion \qq{nachzurüsten}. Hierdurch soll die Ansteuerung der Geräte wesentlich individueller belegbar sein, ohne dass, wie in anderen schon vorhanden Lösungen, (z.B. Logitech Harmony\footnote{ \url{http://www.logitech.com/de-de/harmony-remotes} ; Stand: \today}) eine neue Fernbedienung die Alte ersetzt. Vielmehr soll ein schon vorhandenes Gerät/vorhandene Geräte diese Aufgabe mit übernehmen können.

Zur Umsetzung dieses Vorhabens bietet sich der Mikrocontroller ESP8266 an.
Er ist preiswert und besitzt ein integriertes WLAN-Modul.

\subsection{Ziele}
Ziel des Projektes ist es, eine Schaltung einschließlich dazugehöriger Mikrocontroller-Software zu entwickeln, die folgende Funktionen bietet:

\begin{itemize}
	\item Mikrocontroller muss die Dienste eines Webservers bereitstellen
	\item Kommunikation mit ESP8266 über drahtloses Netzwerk
	\item AccessPoint zur Konfiguration des ESP8266
	\item Einwahl in vorhandenes drahtloses Netzwerk und Speichern der Einwahldaten
	\item Steuerung des ESP8266 über Website
	\item Erfassen, Speichern und Ausgabe von Infrarot-Signalen (38kHz)
\end{itemize}

Alle diese Ziele sind mit Hilfe eines ESP8266 und weiterer Komponenten realisierbar.

\section{Grundlagen}
\subsection{Der ESP8266}
Der ESP8266 ist ein kostengünstiges, programmierbares \ac{SoC} mit einer \acs{WLAN}, \acs{UART}-und \acs{SPI}-Schnittstelle.
Ursprünglich wurde dieser \acs{SoC} von Espressif (daher ESP) entwickelt und verkauft, inzwischen bieten aber verschiedene andere Hersteller ebenfalls Varianten des ESP8266, bzw. dem erweiterten ESP8266EX an.
Diese sind ab einem Stückpreis von ca. 2,00 € z.B. über AliExpress\footnote{\url{http://www.aliexpress.com} ; Stand: \today} verfügbar.

Nachfolgend ist die Spezifikation eines ESP8266 laut Espressif\footnote{\url{http://www.mikrocontroller.net/articles/ESP8266} ; Stand: \today} gelistet:

\begin{itemize}
    \item Wi-Fi Standards: 802.11 b/g/n, Wi-Fi Direct (\acs{P2P}), \acs{soft-AP}
    \item Integrierter TCP/IP Protokollstack
    \item Integrierter \acs{T/R} switch, \acs{balun}, \acs{LNA}, Leistungsverstärker
    \item Integrierte \acs{PLL}s, \acs{DCXO} und Spannungsversorgung
    \item +19.5dBm Ausgangsleistung im 802.11b Modus
    \item Power-Down Leckstrom <10$\mu$A
    \item Integrierte Verbrauchsarme 32-bit CPU (80 MHz)
    \item \acs{SDIO} 1.1/2.0, \acs{SPI}, \acs{UART}
    \item \acs{STBC}, 2×1 \acs{MIMO}
    \item \acs{A-MPDU} \& \acs{A-MSDU} \& 0.4ms Schutzintervalle
    \item Aufwachen und übermitteln von Paketen in < 2ms
    \item Standby Leistungsaufnahme < 1.0mW (\acs{DTIM3})
    \item\acs{VCC}: 3,3V (Nicht 5V tolerant)
    \item Flash Größe: 512kB - 4MB 
\end{itemize}

Verfügbare Flash-Größe, Arbeitsspeicher und CPU-Frequenz sind Modell und Hersteller abhängig. 
Häufig wird die versendete Form innerhalb des online Angebots nicht genauer beschrieben.

In der \autoref{fig:ESP8266} ist ein ESP8266 mit dem dazugehörigen Pinlayout dargestellt.

\begin{figure}[h]
	\centering
	\begin{minipage}{0.45\textwidth}
			\includegraphics[scale=1.5]{Abbildungen/ESP8266A}
	\end{minipage}
	\hfill
	\begin{minipage}{0.45\textwidth} 
			\includegraphics[scale=1.5]{Abbildungen/ESP8266}
	\end{minipage}
	\caption{ESP8266 Platine und Schaltung}
	\label{fig:ESP8266}
\end{figure}

\subsection{Programmierung}
Die Programmierung des Mikrocontrollers kann mit der Arduino \acs{IDE}\footnote{\url{https://www.arduino.cc/} ; Stand: \today} erfolgen. Hierzu muss das Community Package esp8266 installiert sein, welches zusätzliche Header und Funktionen, sowie Compiler und Programmierdefinitionen beinhaltet.
In der \acs{IDE} kann mit einer C/C++-Syntax programmiert werden, wobei der Compiler einige, aber nicht alle Befehlssätze des C++11 Standards unterstützt.

Weiterhin wird der ESP8266 über die \acs{TXD}- und \acs{RXD}-Leitungen mit Hilfe des \acs{UART}-Protokolls programmiert.
Damit der Controller in den Programmiermodus wechselt, muss beim Start der \acs{GPIO}\_0 auf Masse liegen (Controller startet im Programmiermodus).
Anschließend kann die durch Cross-Compiling erzeugte Binary von der Arduino IDE an den ESP8266 übertragen werden.
Für die Programmierung bieten sich beispielsweise sogenannte USB-to-\acs{UART}-Programmierkabel an (z.B. der PL2303).
Hierbei ist zu beachten, dass die Spannungsversorgung am Ausgang der meisten Programmierkabel 5V und nicht 3.3V beträgt. Der Mikrocontroller ist nicht 5V tolerant.

\subsection{Funktionsweise Fernbedienung}

Eine übliche Infrarot-Fernbedienung sendet ein Signal welches bei den meisten Herstellern von Fernsehern auf eine Trägerfrequenz von 36 - 38 kHz moduliert ist.
In diesem Projekt wird eine Fernbedingung mit 38kHz realisiert. Bei genügend hoher Toleranz können hiermit aber auch Empfänger gesteuert werden, welche ein 36 kHz Signal erwarten.
Es ist jedoch wichtig sich auf eine Frequenz festzulegen, da die Frequenz Einfluss auf Bauteile und Programmierung hat.

Über eine \ac{PWM} auf die Trägerfrequenz wird das codierte Signal übertragen.
Jede Taste auf einer Fernbedingung besitzt eine eigene Codierung.
Weiterhin existieren verschiedene Protokolle zur Codierung von Signalen.
Die Protokolle werden in diesem Projekt vernachlässigt, da die Codierung im Wesentlichen nur kopiert und gespeichert wird.

Damit das Infrarotsignal empfangen werden kann, wird ein entsprechender Empfänger benötigt.
Solche Empfänger sind als integrierte Schaltkreise verfügbar, welche alle benötigten Funktionen in einem Bauteil vereinen.
Ein Infrarotempfänger besteht aus einer Fotodiode, einem geregeltem Verstärker, welcher entsprechend der Frequenz arbeitet (zum Beispiel 38kHz) und einem Demodulator.
Der Demodulator sendet das digital codierte Signal an den Mikrocontroller.
Weiterhin besitzt der integrierte Schaltkreis einen Bandpassfilter um Störungen von anderen Frequenzen als der gewünschten Frequenz zu vermeiden.
Meist sind Fotodiode und Schaltkreis in einem Kunststoffgehäuse integriert.
Wir haben uns für den TSOP4838\footnote{Datenblatt: \url{http://mkpochtoi.narod.ru/TSOP4836_ds.pdf} Stand: \today} entschieden, da dieser auf 38 kHz abgestimmt ist. Weitere Trägerfrequenzen sind in gleicher Bauform erhältlich.

Auf die konkreten Details zu \acs{PWM} wird in dieser Dokumentation nicht eingegangen.

\section{Realisierung}
Nachdem die Grundlagen für das Projekt betrachtet wurden, soll nun die konkrete Realisierung des Projektes verdeutlicht werden.
\subsection{Komponentenliste}
Für die Realisierung des Projektes wurden folgende Komponenten genutzt:
\begin{itemize}
	\item Mikrocontroller: ESP8266
	\item linearer Spannungsregler für 3,3V: LM1117-3V3
	\item Zwei Kondensatoren: 10 $\mu$F, 100nF
	\item Infrarotdiode: 1,5V / 80mA
	\item Vorwiderstand für Diode: 22 Ohm
	\item 2 poliger Umschalter
	\item Taster
	\item Infrarotfotodiode: TSOP4838 für die Trägerfrequenz 38kHz
	\item USB-to-UART-Programmierer: PL2303
\end{itemize}


\subsection{Aufbau}
Die \autoref{fig:Schalplan} zeigt den Schaltplan für die Realisierung einer Infrarotfernbedingung inklusive Empfänger.

\begin{figure}[h]
	\centering
	\includegraphics[scale=1.4]{Abbildungen/ESP8266_Schaltplan}
	\caption{Schaltplan}
	\label{fig:Schalplan}
\end{figure}

Für die Programmierung des Mikrocontrollers wird ein USB-to-\acs{UART}-Programmierer (PL2303) verwendet.
Dieser liefert an \acs{VCC} 5V und 3,3V an TXD sowie RXD.
Wird er ohne PC betrieben kann er über ein einfaches 5V USB Netzteil auch als Zuleitung verwendet werden. Die \acs{TXD} und \acs{RXD}-Leitungen liegen dann auf 0V.
Der ESP8266 selbst benötigt 3,3V und ist nicht 5V tolerant, sodass ein Spannungsregler verwendet wird um 3,3V für den Mikrocontroller zur Verfügung zu stellen.
Als Spannungsregler bietet sich zum Beispiel der LM1117-3V3\footnote{\url{http://www.ti.com/lit/ds/symlink/lm1117.pdf} ; Stand: \today} (verwendet) oder ein L7805 an.
Zur Spannungsstabilisierung wird dem Regler ein 100 nF Kondensator vor und ein 10$\mu$F-Kondensator nachgeschaltet.

Die Pins \acs{VCC} und \acs{CHPD} werden mit der Spannungsquelle 3,3V verbunden und \ac{GND} entsprechend auf Masse gelegt.
Der \acs{RST}-Pin wird mit einem Taster verbunden, welcher bei betätigen diesen Pin auf Masse zieht (interner Pull-Up) und somit den Mikrocontroller zurücksetzt.

Das ESP8266-Modul besitzt 2 Digitale Ein- beziehungsweise Ausgänge. \acs{GPIO}\_0 wird als Ausgang und \acs{GPIO}\_2 als Eingang genutzt.
Wie diese Pins arbeiten wird bei der Programmierung festgelegt.

An dem {GPIO}\_0 ist ein Schalter angeschlossen.
Dieser dient dazu zwischen dem Programmiermodus und dem normalen Modus zu Wechseln.
Der Pin ist Low-Aktiv, sodass für den normalen Betrieb nach einem Reset immer 3,3V anliegen müssen.
Für das Programmieren wird dieser Pin auf Masse gezogen in dem der Schalter entsprechend gesetzt und ein Reset ausgeführt wird.
Andernfalls ist der Schalter mit einer Infrarotdiode (1,5V, 80mA) inklusive Vorwiederstand(22 Ohm) verbunden.
Die Diode und der Widerstand sind mit der 3,3V Spannungsquelle verbunden.
Diese Diode dient später dazu die Infrarotsignale zu senden.

Der \acs{GPIO}\_2 wird mit dem Infrarotempfänger verbunden und dient zur Erfassung der Infrarot-Signale.
Der Infrarotempfänger wird entsprechend mit der 3,3V Spannungsquelle und der Masse verbunden.

Aufgrund dessen, dass Senden und Empfangen nicht gleichzeitig möglich sind, bedarf es keiner Abschirmung zwischen Sende und Empfangsdiode. Es wird über die Programmierung sichergestellt, dass der Sender während des erwarteten Erfassen von Signalen nicht leuchtet (Ausgang liegt auf High).

Die \autoref{fig:Steckplatine} zeigt den Aufbau auf einer Steckplatine.

\begin{figure}[h]
	\centering
	\includegraphics[scale=1]{Abbildungen/ESP8266_Steckplatine}
	\caption{Steckplatine}
	\label{fig:Steckplatine}
\end{figure}~\\

\subsection{Implementierung}
Ein Programm für den ESP8266 besteht aus 2 Teilen.
Diese 2 Teile sind jeweils definiert durch die Funktionen \texttt{setup()} (Initialisierung des Mikrocontrollers) und \texttt{loop()} (Schleife).

In der Funktion \texttt{setup} werden Befehle ausgeführt, welche beim Starten des Mikrocontrollers aufgerufen werden sollen.
Zum Beispiel ist es möglich im Setup eine Verbindung zum einem drahtlosen Netzwerk herzustellen und Serverfunktionalität bereitzustellen.
Nach Abschluss des Setups gelangt der Mikrocontroller in die \texttt{loop}-Funktion.
Diese Funktion wird immer wieder aufgerufen und entsprechende Befehle in der Funktion ausgeführt.
Zum Beispiel wird an dieser Stelle geprüft ob ein Client eine Anfrage an den Mikrocontroller gesendet hat.
Falls ja, werden die entsprechenden Funktionen ausgeführt.

Der ESP8266 besitzt einen \acs{EEPROM} und dessen Größe ist Modell abhängig (512kB - 4MB).
In der derzeitigen Implementierung wird nur ein Sektor (4096Bytes) verwendet.
Dies ist begründet mit der Verwendung des ESP8266 Community Package inklusive Header, welches derzeit keine größere Größe vorsieht.

Dieser Speicher wird in der Implementierung wie folgt genutzt: 
\begin{itemize}
	\item 0 - 96 = WLAN-Daten
	\item 97 - 149 = Vorgehalten für weitere Einstellungen
	\item 150 = Stationsanzahl bzw. Anzahl der gespeicherten Infrarotcodes
	\item 151 - 4095 = Stationen
\end{itemize}
Die tatsächliche Länge von Infrarotcodes ist abhängig von den verwendeten Protokollen und variieren in der Anzahl des benötigten Speichers.
In der \autoref{fig:EEPROM} ist die Aufteilung des Speichers Schematisch dargestellt.
Eine Station besteht aus der Bezeichnung (Namen) der Station und die dazugehörige Anzahl an Buchstaben sowie die dazugehörigen IR-Codes (Zeiten) mit der Anzahl an Werten.
\begin{figure}[!ht]
\centering
\begin{tikzpicture}
%Erster Block
\draw (0,0) -- (1,0) (1,1) --  (0,1) -- (0,0);
\draw (2,0) -- (3,0)  -- (3,1) --  (2,1);
\draw[loosely dotted] (2,1) -- (1,1);
\draw[loosely dotted] (2,0) -- (1,0);
\node[text width=4cm] at (2.4,1.2) {0};
\node[text width=4cm] at (4.3,1.2) {96};
\node[text width=4cm] at (2.2,0.45) {WiFi-Settings};
%Zweiter Block
\draw (4,0) -- (5,0) -- (5,1) --  (4,1) -- (4,0);
\draw (5,0) -- (6,0) (6,1) --  (5,1) -- (5,0);
\draw (7,0) -- (8,0)  -- (8,1) --  (7,1);
\draw[loosely dotted] (6,1) -- (7,1);
\draw[loosely dotted] (6,0) -- (7,0);
\node[text width=4cm] at (6.2,1.2) {150};
\node[text width=4cm] at (7.2,1.2) {151};
\node[text width=4cm] at (9.2,1.2) {4095};

\node[text width=4cm] at (6.05,0.45) {\scriptsize Anzahl};
\node[text width=4cm] at (7.7,0.45) {IR-Codes};
% Detailansicht von IR-Codes

\draw (1,-2) -- (2,-2) -- (2,-1) --  (1,-1) -- (1,-2);
\draw (2,-2) -- (3,-2)  (3,-1) --  (2,-1) -- (2,-2);
\draw (4,-2) -- (5,-2)  -- (5,-1) --  (4,-1);
\draw[loosely dotted] (3,-1) -- (4,-1);
\draw[loosely dotted] (3,-2) -- (4,-2);
\node[text width=4cm] at (3.05,-1.55) {\scriptsize Anzahl};
\node[text width=4cm] at (4.2,-1.55) {\scriptsize ~ ~ ~ Name \\ (1 Byte / Zeichen)};
\draw (5,-2) -- (6,-2) -- (6,-1) --  (5,-1) -- (5,-2);
\draw (6,-2) -- (7,-2)  (7,-1) --  (6,-1) -- (6,-2);
\draw (8,-2) -- (9,-2)  -- (9,-1) --  (8,-1);
\draw[loosely dotted] (7,-1) -- (8,-1);
\draw[loosely dotted] (7,-2) -- (8,-2);
\node[text width=4cm] at (7.05,-1.55) {\scriptsize Anzahl};
\node[text width=4cm] at (8.6,-1.55) {\scriptsize ~ ~ Zeiten \\ (2 Bytes / Int)};
\draw[loosely dotted] (9,-1) -- (10,-1);
\draw[loosely dotted] (9,-2) -- (10,-2);

%Verbindungen
\draw (5,0) -- (1,-1);
\draw (8,0) -- (10,-1);

\end{tikzpicture}
\caption{Aufbau des Speichers}
\label{fig:EEPROM}
\end{figure}


\subsubsection{Setup}
%TODO Okay so?

% Define block styles
\tikzstyle{decision} = [diamond, draw, 
    text width=5.5em, text badly centered, node distance=2.7cm, inner sep=0pt]
\tikzstyle{block} = [rectangle, draw, 
    text width=6em, text centered, rounded corners, minimum height=4em]
\tikzstyle{line} = [draw, -latex']
\tikzstyle{cloud} = [draw, ellipse, node distance=3.2cm,
    minimum height=3em]

\begin{figure}[!ht]
\centering 
\begin{tikzpicture}[node distance = 2cm, auto, scale=0.8, every node/.style={scale=0.8}]
    % Place nodes
    \node [block] (init) {Start};
    \node [cloud, left of=init] (setup) {Setup( )};
    \node [block, below of=init, node distance=2.2cm] (confPin) {Konfiguriere GPIOs};
    \node [block, below of=confPin, node distance=2.2cm] (readIR) {Lese Infrarot-Daten aus Speicher};
    \node [block, below of=readIR, node distance=2.2cm] (readWLAN) {Lese WLAN-Daten aus Speicher};
    \node [block, below of=readWLAN, node distance=2.2cm] (connectWLAN) {Verbinde mit WLAN};
    \node [decision, below of=connectWLAN] (decide) {Erfolgreich?};
   	\node [block, left of=decide, node distance=4cm] (startAP) {Starte als AP};
    \node [block, below of=decide, node distance=3cm] (mDNS) {Starte \acs{mDNS}};
 	\node [block, below of=mDNS, node distance=2.2cm] (launch) {Starte Server};
    \node [block, below of=launch, node distance=2.2cm] (ende) {Ende};
    \node [cloud, right of=ende] (loop) {Loop( )};
    % Draw edges
    \path [line] (init) -- (confPin);
    \path [line] (confPin) -- (readIR);
    \path [line] (readIR) -- (readWLAN);
    \path [line] (readWLAN) -- (connectWLAN);
    \path [line] (connectWLAN) -- (decide);
	\path [line] (decide) -- node [near start] {Nein} (startAP);
    \path [line] (startAP) |- (launch);
    \path [line] (decide) -- node {Ja}(mDNS);
    \path [line] (mDNS) -- (launch);
    \path [line] (launch) -- (ende);
    \path [line] (setup) -- (init);
    \path [line] (ende) -- (loop);
\end{tikzpicture}

\caption{Programmablaufplan für Setup()}
\label{PAP:Setup}
\end{figure}

Die \autoref{PAP:Setup} zeigt den Programmablaufplan für den Aufruf der Setup-Funktion.
Beim Starten des ESP8266 führt die Setup-Funktion folgende Schritte aus:\\
Zunächst wird \acs{GPIO}\_0 als Ausgabe und \acs{GPIO}\_2 als Eingabe-Pin festgelegt.
Im Anschluss werden gespeicherte Infrarotcodes, auch als Stationen bezeichnet, aus dem \acs{EEPROM} gelesen.

Nachdem die Infrarotcodes gelesen wurden, werden nun die drahtlosen Netzwerkdaten aus dem \acs{EEPROM} des ESP8266 gelesen.
Mit diesen Daten wird versucht sich in das drahtlose Netzwerk einzubinden.
Falls die Verbindung fehlschlägt, zum Beispiel, weil ein falsches Passwort gespeichert wurde, dann startet der Mikrocontroller als AccessPoint.
Im Anschluss werden Funktionen ausgeführt, welche Serverfähigkeiten bereitstellen und die Setup-Funktion ist fertig.
Nun kann ein drahtloses Netzwerkfähiges Gerät sich mit dem ESP8266 verbinden und über die IP-Adresse 192.168.4.1 im Browser auf den ESP8266 zugreifen und konfigurieren.

Sollte die Verbindung mit einem drahtloses Netzwerk erfolgreich sein, so wird zusätzlich noch der \acs{mDNS}-Dienst ausgeführt.
Dies ist ein konfigurationsloses Verfahren um lokale Netzwerkadressen als Namen bereitzustellen.
Dieses Verfahren muss explizit vom entsprechendem System unterstützt werden.
Es kann anschließend über \url{iot_remote.local} oder die zugewiesene IP-Adresse auf den ESP8266 zugegriffen werden.

Den Zugriff und die Darstellung der Website des ESP8266 wird in der Loop-Funktion geregelt.
\subsubsection{Loop}
%TODO Okay so?
\begin{figure}[!ht]
\centering 
\begin{tikzpicture}[node distance = 2cm, auto, scale=0.8, every node/.style={scale=0.8}]
    % Place nodes
    \node [block] (init) {Start};
    \node [cloud, left of=init] (loopStart) {Loop( )};
    \node [decision, below of=init] (decide) {Anfrage Client ?};
    \node [block, below of=decide, node distance=3cm] (end) {Ende};
    \node [cloud, right of=end] (loopEnd) {Loop( )};
    \node [block, left of=decide, node distance=4cm] (work) {Führe Anfrage aus};
    \node [block, below of=work, node distance=3cm] (site) {Stelle Website dar};
    % Draw edges
    \path [line] (init) -- (decide);
    \path [line,dashed] (loopStart) -- (init); 
    \path [line] (decide) -- node [near start] {Nein} (end);
    \path [line,dashed] (end) -- (loopEnd);
    \path [line] (decide) -- node {Ja}(work);
    \path [line] (work) -- (site);
     \path [line] (site) -- (end);
\end{tikzpicture}
\caption{Programmablaufplan für Loop()}
\label{PAP:Loop}
\end{figure}

Die \autoref{PAP:Loop} zeigt den stark vereinfachten Programmablaufplan für den Aufruf der Loop-Funktion.
Sobald ein Endgerät über die IP-Adresse oder über die Adresse \url{iot_remote.local}\footnote{Nicht im AccessPoint-Modus}\footnote{Setzt \acs{mDNS}-Unterstützung voraus} auf den ESP8266 zugreift, wird die Anfrage entsprechend bearbeitet und anschließend die darzustellende Seite an das Endgerät gesendet.
Die \autoref{fig:Start} zeigt diese Startseite der Anwendung beim Aufruf mittels eines Browsers.
\begin{figure}[!ht]
	\centering
	\includegraphics[scale=0.3]{Abbildungen/Start}
	\caption{Startseite der Anwendung}
	\label{fig:Start}
\end{figure}
Anschließend führt der ESP8266 die Schleife weiter aus.
Ein Klick auf eine Schaltfläche führt die entsprechende Funktionalität auf dem ESP8266 aus.

Die Schaltfläche \qq{Settings} führt zur Konfigurationsseite für drahtlose Netzwerke.
In der \autoref{fig:WIFI} wird die Konfigurationsmaske für drahtlose Netzwerkdaten dargestellt.
\begin{figure}[!ht]
	\centering
	\includegraphics[scale=0.3]{Abbildungen/WIFI}
	\caption{Konfigurationsmaske für drahtlose Netzwerkdaten}
	\label{fig:WIFI}
\end{figure}
Bei Klick auf \qq{Submit} werden die angegeben Netzwerkdaten in dem \acs{EEPROM} des ESP8266 gespeichert und bei einem Neustart verwendet um eine Verbindung zum drahtlosen Netzwerk herzustellen.
Weiterhin ist es mit Klick auf \qq{clear all stations} möglich alle gespeicherten Infrarotcodes zurückzusetzen.

Wird auf der Startseite \qq{Record New} gewählt, dann können Infrarotcodes aufgenommen, bezeichnet und abgespeichert werden.
Das Ergebnis der Aufnahme wird bei \autoref{fig:Show} gezeigt. Der Code wird als Liste von Zahlen gespeichert, welche den Zeitpunkt des Wechsels von Low zu High oder von High zu Low beschreiben. Hierbei gibt jede Zahl den Takt in der Trägerfrequenz seid dem ersten empfangenen High an.
%TODO Stimmt diese Aussage?
\begin{figure}[!ht]
	\centering
	\includegraphics[scale=0.3]{Abbildungen/Show}
	\caption{Maske zum Speichern der aufgenommenen Infrarotcodes}
	\label{fig:Show}
\end{figure}\\
Angelegte Infrarotcodes inklusive Bezeichnung werden auf der Startseite als Schaltflächen dargestellt.
Werden diese angeklickt, so wird der gespeicherte Infrarotcode gesendet und zur Startseite zurückgekehrt.
In der \autoref{fig:Start} sind solche Tasten mit der Bezeichnung \qq{On} und \qq{Off} dargestellt, welche zum Beispiel als An- und Ausschalter eines Gerätes dienen.

\subsubsection{Infrarotsignale Empfangen}
%TODO mit Pseudocode?
In diesem Projekt wird die Fotodiode TSOP4838 als Empfänger für Signale auf der Trägerfrequenz 38kHz verwendet.
Damit Infrarotsignale vom ESP8266 aufgenommen werden, muss sich die Anwendung im Aufnahmemodus für Infrarotcodes befinden.
Hierbei wird der \acs{GPIO}\_2 überwacht, welcher die dekodierten Signale des TSOP4838 ließt.

Realisiert wird das ganze in dem am \acs{GPIO}\_2 alle 26.3 $\mu$s das angelegte Signal gelesen und mit dem vorherigen verglichen wird. Ist es unterschiedlich so wird die Taktnummer gespeichert. Es werden dabei 10.000 Lesevorgänge durchgeführt, sodass eine maximal länge von 263 ms pro IR-Code nicht überschritten werden darf.
Dies bietet den Vorteil, dass die Aufnahme zeitlich definiert ist, da der ESP8266 während dieser Zeit keine HTTP-Anfragen bearbeiten kann.

\subsubsection{Infrarotsignale Senden}
Damit Infrarotsignale mit einer Trägerfrequenz von 38kHz gesendet werden, bedarf es einem Trick in der Programmierung um dies zu realisieren.
Dies geschieht in dem der digitale Ausgang \acs{GPIO}\_0 zwischen einer logischen 1 und 0 alterniert um das modulierte Signal auf der Trägerfrequenz zu emulieren.

Dies kann mit Hilfe eines $\mu$s-delay Befehl realisiert werden.
Jeder Puls ist 1/38000 Sekunden (26,3 $\mu$s) lang. 
Gerundet auf 26 $\mu$s ist die Modulationsfrequenz realisierbar mit dem Alternieren von 1 und 0, wobei ein Takt aus 13$\mu$s 1 und 13$\mu$s 0 besteht.
Die restlichen 0.3 $\mu$s werden ca. von den Ansteuerungsfunktionen für den Wechsel zwischen 1 und 0 sowie der Schleife verwendet.

\section{Fazit}
%TODO Herr Bastian mag es, wenn ALLES gelernte in das Fazit / Zusammenfassung einfließt-> Fazit darf ruhig mehrere Seiten lang sein!!
%TODO Bitte ergänze das hier bitte mit @ Eike
%TODO Vervollständigen

Im Verlaufe des Projektes ist eine, über ein drahtloses Netzwerk steuerbare, Mikro-controller-Anwendung entstanden, welche in einem heimischen Netzwerk verwendet werden kann.
Dabei können WLAN- und Infrarotdaten gespeichert werden.
Außerdem können gespeicherte Infrarotsignale gesendet werden, um so zum Beispiel einen Umschaltvorgang eines Fernsehers auszulösen.
Die ganze Realisierung ist im Vergleich zu einer Logitech MyHarmony Universalfernbedingung sehr kostengünstig.

Problematisch in einem heimischen Netzwerk ist, dass die IP-Adresse im Netzwerk typischerweise dynamisch vergeben wird.
Daher muss die IP-Adresse zuvor ermittelt werden, um per Browser auf die Website des ESP8266 zugreifen zu können.
Durch die mDNS Unterstützung kann einfach über \url{iot_remote.local} zugegriffen werden, sodass die IP-Adresse nicht bekannt sein muss.
Im AcessPoint-Modus wird immer über die IP-Adresse \url{192.168.4.1} zugegriffen.

Folgende Fertigkeiten wurden im Verlaufe des Projektes erworben:
\begin{itemize}
	\item Programmierung in einer Arduino \acs{IDE} inklusive entsprechender Konfiguration der \acs{IDE}
	\item Planen und Programmieren einer Firmware eines ESP8266 in C/C++ unter Verwendung von gegebenen Header-Dateien.
	\item Umgang mit dem ESP8266 und den Schaltungs-Design-Beschränkungen (GPIO\_0 und GPIO\_2)
	\item Erworbenes Verständnis für die Funktionsweise von Infrarotsendern und Empfängern, \acs{PWM}
\end{itemize} 

Es ist in der gewählten Realisierung nicht möglich die Stromsparmaßnahmen des Mikrocontrollers zu verwenden, da hierbei die Verbindung zum drahtlosen Netzwerk unterbrochen werden. 
Der Mikrocontroller kann dann nicht mehr auf Anfragen aus dem Netzwerk reagieren und dadurch nicht daraus geweckt werden.

Der Quellcode für das Projekt ist auf GitHub verfügbar unter:\\ \url{https://github.com/Ava-chan/IOT_IR_Remote}\footnote{Stand: \today}

\pagebreak

\begin{appendix}
\pagestyle{empty}	
\section{Abkürzungsverzeichnis}

\begin{acronym}
\acro{A-MPDU}{\textbf{A}ggregated \textbf{M}ac \textbf{P}rotocol \textbf{D}ata \textbf{U}nit}{ - Eine Funktion im Wlan-Standard 802.11e und 802.11n um bei einer einzelne Übertragung mehrere Pakete zu senden}
\acro{A-MSDU}{\textbf{A}ggregated \textbf{M}ac \textbf{S}ervice \textbf{D}ata}{ - Eine Funktion im Wlan-Standard 802.11e und 802.11n um bei einer einzelne Übertragung mehrere Pakete zu senden}
\acro{balun}{\textbf{bal}anced-\textbf{un}balanced}{ - Wandler zwischen symmetrischen und unsymmetrischen Leistungssystemen}
\acro{CHPD}{\textbf{C}hip \textbf{P}ower \textbf{D}own}{ -Der Chip kann mit Hilfe dieses Pins vollständig heruntergefahren werden}
\acro{DCXO}{\textbf{D}igitally \textbf{C}ontrolled \textbf{O}scillator}{ - digital gesteuerter Quarzoszillator}
\acro{DTIM3}{\textbf{D}elivery \textbf{T}raffic \textbf{I}ndication \textbf{M}essage}{- regelmäßiger Broadcast des Access Point um Wireless-Stationen aufzuwecken}
\acro{EEPROM}{\textbf{E}lectrically \textbf{E}rasable \textbf{P}rogrammable \textbf{R}ead-\textbf{O}nly \textbf{M}emory}{ - elektrisch löschbarer programmierbarer Nur-Lese-Speicher}
\acro{GND}{Ground}{ - Masse}
\acro{GPIO}{\textbf{G}eneral \textbf{P}urpose \textbf{I}nput \textbf{O}utput}{ - Durch logische Programmierung kann PIN Verhalten als Eingang oder Ausgang definiert werden}
\acro{IDE}{\textbf{I}ntegrated \textbf{D}evelopment \textbf{E}nvironment}{ - Entwicklungsumgebung zum Zwecke der Programmierung}
\acro{IoT}{\textbf{I}nternet \textbf{o}f \textbf{T}hings}{ - \qq{intelligente Gegenstände} unterstützen Menschen unmerklich}
\acro{LNA}{\textbf{L}ow \textbf{N}oise \textbf{A}mplifier}{ - Rausch armer Verstärker}
\acro{mDNS}{\textbf{m}ulticast \textbf{DNS}}{- Ist ein Konfigurationsloses Verfahren um Netzwerkadressen als Namen zu IP-Adressen aufzulösen, dieses Verfahren muss vom Client-System unterstützt werden}
\acro{MIMO}{\textbf{M}ultiple \textbf{I}nput \textbf{M}ultiple \textbf{O}utput}{ - Nutzung mehrerer Sende- und Empfangsantennen}
\acro{P2P}{\textbf{P}eer to \textbf{P}eer}{ - Direkte Endgerät zu Endgerät Kommunikation}
\acro{PLL}{\textbf{P}hase-\textbf{L}ocked \textbf{L}oop}{ - Damit ein konstantes abgeleitetes Signal zwischen einem äußeren Referenzsignal und Oszillator entsteht, wird solch eine Phasenregelschleife verwendet} 
\acro{PWM}{\textbf{P}uls-\textbf{W}eiten-\textbf{M}odulation}
\acro{RST}{Reset}{ -Setzt den Controller zurück, sodass dieser einen Neustart durchführt}
\acro{RXD}{\textbf{R}eceive \textbf{D}ata}{ - Eingehende Leitung für Daten}
\acro{soft-AP}{\textbf{soft}ware enabled \textbf{A}ccess \textbf{P}oint}{ - ein über Software erstellter Access Point, welcher auf einem Gerät erzeugt wird welches nicht speziell als Router gedacht ist}
\acro{SDIO}{\textbf{S}ecure \textbf{D}igital \textbf{I}nput \textbf{O}utput}{ - SD-Interface um Peripheriegeräte (z.B. Webcam, GPS, Ethernet, ...) anzusprechen}
\acro{SPI}{\textbf{S}erial \textbf{P}eripheral \textbf{I}nterface}{ - Synchroner serieller Datenbus}
\acro{SoC}{\textbf{S}ystem-\textbf{o}n-a-\textbf{C}hip}{ - Integration aller Systeme in einem Chip / einem Bauteil}
\acro{STBC}{\textbf{S}pace \textbf{T}ime \textbf{B}lock \textbf{C}oding}{ - Übertragungsverfahren in Funknetzwerken um mehrfach Daten über mehrere Antennen verteilt zu versenden}
\acro{T/R}{\textbf{T}ransmit / \textbf{R}eceive}{ - Versenden und Empfangen von Signalen}
\acro{TXD}{\textbf{T}ransmit \textbf{D}ata}{ - Ausgehende Leitung für Daten}
\acro{UART}{\textbf{U}niversal \textbf{A}synchronous \textbf{R}eceiver \textbf{T}ransmitter} { - digitale serielle Schnittstelle mit einem speziellen Protokoll}
\acro{VCC}{\textbf{V}oltage at the \textbf{C}ommon \textbf{C}ollector} { - Versorgungsspannung}
\acro{WLAN}{\textbf{W}ireless \textbf{L}ocal \textbf{A}rea \textbf{N}etwork}{ - engl. auch WiFi - kabellose Netzwerkverbindung}
\end{acronym}
\end{appendix}

\end{document}